
@Article{ijgi9060355,
AUTHOR = {Guldåker, Nicklas},
TITLE = {Geovisualization and Geographical Analysis for Fire Prevention},
JOURNAL = {ISPRS International Journal of Geo-Information},
VOLUME = {9},
YEAR = {2020},
NUMBER = {6},
ARTICLE-NUMBER = {355},
URL = {https://www.mdpi.com/2220-9964/9/6/355},
ISSN = {2220-9964},
ABSTRACT = {Swedish emergency services still have relatively limited resources and time for proactive fire prevention. As a result of this, there is an extensive need for strategic working methods and knowledge to take advantage of spatial analyses. In addition, decision-making based on visualizations and analyses of their own collected data has the potential to increase the validity of strategic decisions. The objective of this paper is to critically examine how some different geovisualization techniques—point data, kernel density and choropleth mapping—actively can complement each other and be applied in fire preventive work. The results show that each technique itself has limitations, but that, in combination, they increase the scope for interpretation and the possibilities of targeting different forms of preventive measures. The investigated geovisualization techniques facilitate various forms of fire prevention such as identifying which areas to prioritize for outreach, home visits, identification and targeting of different risk groups and customized information campaigns about certain types of fires in risk-prone areas. Furthermore, fairly simple mapping techniques can be utilized directly to evaluate incident reports and increase the quality of geocoded fire incidents. The study also shows how some of these techniques can be applied when analyzing residential fire incidents and their relation to underlying structural and socio-economic factors as well as spatio-temporal dimensions of fire incident data. The spatial analyses and supporting maps can help find and predict risk areas for residential fires or be used directly to formulate hypotheses on fire patterns. The generic functionality of the visualization methods makes them also useful for visual analysis of other types of incidents, such as reported crimes and accidents. Finally, the results are applicable to a work process adapted to the Swedish legislation on confidential data.},
DOI = {10.3390/ijgi9060355}
}



