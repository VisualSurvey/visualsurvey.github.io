@article{MACHADO201784,
title = {Visual soccer match analysis using spatiotemporal positions of players},
journal = {Computers & Graphics},
volume = {68},
pages = {84-95},
year = {2017},
issn = {0097-8493},
doi = {https://doi.org/10.1016/j.cag.2017.08.006},
url = {https://www.sciencedirect.com/science/article/pii/S0097849317301358},
author = {Vinicius Machado and Roger Leite and Felipe Moura and Sergio Cunha and Filip Sadlo and João L.D. Comba},
keywords = {Sports analytics, Visual analytics},
abstract = {Soccer is a fascinating sport that captures the attention of millions of people in the world. Professional soccer teams, as well as the broadcasting media, have a deep interest in the analysis of soccer matches. Statistical summaries are the most widely used approach to describe a soccer match. However, they often fail to capture the evolution and changes of strategies that happen during a game. In this work, we present visual designs to help understanding a soccer match from the spatiotemporal position of players. We receive as input the coordinates of each player throughout the match, as well as the associated events. We present a pixel-oriented layout that summarizes the changes of player positions and tactical schemes during the match. Also, we revisit a technique used for flow analysis to help us identify where does a player move from a given region in the field. We developed our approach in conjunction with colleagues from physical education with experience in soccer analysis, as well as experts on soccer data extraction. We demonstrate the utility of our approach in several match situations, and provide the feedback given by the experts.}
}