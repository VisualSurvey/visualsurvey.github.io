@inproceedings{10.1145/1842993.1843000,
author = {Forlines, Clifton and Wittenburg, Kent},
title = {Wakame: Sense Making of Multi-Dimensional Spatial-Temporal Data},
year = {2010},
isbn = {9781450300766},
publisher = {Association for Computing Machinery},
address = {New York, NY, USA},
url = {https://doi.org/10.1145/1842993.1843000},
doi = {10.1145/1842993.1843000},
abstract = {As our ability to measure the world around us improves, we are quickly generating massive quantities of high-dimensional, spatial-temporal data. In this paper, we concern ourselves with datasets in which the spatial characteristics are relatively static but many dimensions prevail and data is sampled over different time periods. Example applications include building energy management and HVAC unit diagnostics. We present methods employed in our Wakame visualization system to support such tasks as discovering anomalies and comparing performance across multiple time series. Novel methods include animated transitions that relate data in spatially located 3D views with conventional 2D graphs. Additionally, several components of our prototype employ analytics to guide the user to "interesting" portions of the dataset.},
booktitle = {Proceedings of the International Conference on Advanced Visual Interfaces},
pages = {33–40},
numpages = {8},
keywords = {infovis, multi-dimensional data, visual analytics, spatial-temporal data, radar graph},
location = {Roma, Italy},
series = {AVI '10}
}