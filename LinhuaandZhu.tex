@article{ref1,
title="A GIS-based Bayesian approach for analyzing spatial-temporal patterns of intra-city motor vehicle crashes",
journal="Journal of transport geography",
year="2007",
author="Li, Linhua and Zhu, Li and Sui, Daniel Z.",
volume="15",
number="4",
pages="274-285",
abstract="This paper develops a GIS-based Bayesian approach for intra-city motor vehicle crash analysis. Five-year crash data for Harris County (primarily the City of Houston), Texas are analyzed using a geographic information system (GIS), and spatial-temporal patterns of relative crash risks are identified based on a Bayesian approach. This approach is used to identify and rank roadway segments with potentially high risks for crashes so that preventive actions can be taken to reduce the risks in these segments. Results demonstrate the approach is useful in estimating the relative crash risks, eliminating the instability of estimates while maintaining overall safety trends. The 3-D posterior risk maps show risky roadway segments where safety improvements need to be implemented. Results of GIS-based Bayesian mapping are also useful for travelers to choose relatively safer routes.<p />",
language="",
issn="0966-6923",
doi="10.1016/j.jtrangeo.2006.08.005",
url="http://dx.doi.org/10.1016/j.jtrangeo.2006.08.005"
}