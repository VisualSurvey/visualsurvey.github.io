@article{ANDRIENKO2003503,
title = {Exploratory spatio-temporal visualization: an analytical review},
journal = {Journal of Visual Languages & Computing},
volume = {14},
number = {6},
pages = {503-541},
year = {2003},
note = {Visual Data Mining},
issn = {1045-926X},
doi = {https://doi.org/10.1016/S1045-926X(03)00046-6},
url = {https://www.sciencedirect.com/science/article/pii/S1045926X03000466},
author = {Natalia Andrienko and Gennady Andrienko and Peter Gatalsky},
abstract = {Current software tools for visualization of spatio-temporal data, on the one hand, utilize the opportunities provided by modern computer technologies, on the other hand, incorporate the legacy from the conventional cartography. We have considered existing visualization-based techniques for exploratory analysis of spatio-temporal data from two perspectives: (1) what types of spatio-temporal data they are applicable to; (2) what exploratory tasks they can potentially support. The technique investigation has been based on an operational typology of spatio-temporal data and analytical tasks we specially devised for this purpose. The result of the study is a structured inventory of existing exploratory techniques related to the types of data and tasks they are appropriate for. This result is potentially helpful for data analysts—users of geovisualization tools: it provides guidelines for selection of proper exploratory techniques depending on the characteristics of data to analyze and the goals of analysis. At the same time the inventory as well as the suggested typology of tasks could be useful for tool designers and developers of various domain-specific geovisualization applications. The designers can, on the one hand, see what task types are insufficiently supported by the existing tools and direct their creative activities towards filling the gaps, on the other hand, use the techniques described as basic elements for building new, more sophisticated ones. The application developers can, on the one hand, use the task and data typology in the analysis of potential user needs, on the other hand, appropriately select and combine existing tools in order to satisfy these needs.}
}